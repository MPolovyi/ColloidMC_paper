\documentclass[12pt,a4paper]{article}
\usepackage[utf8]{inputenc}
\usepackage{amsmath}
\usepackage{amsfonts}
\usepackage{amssymb}
\usepackage{xcolor}
\usepackage[left=2cm,right=2cm,top=2cm,bottom=2cm]{geometry}
\author{Maksym Polovyi}
\title{Colloid MC}
\begin{document}
\section{Model}

\textcolor{red}{In this section we provide theoretical description of the model. Computational aspects will be highlighted in the following section, and the results will be elaborated on in the Results section.}

We consider colloidal particles with permanent dipole moment $\mu \vec{m}$, where $|\vec{m}| = 1$. While considering a three-dimensional system, we constrain particle positions to a one-dimensional line. However, particles can explore full three dimensional orientation space. For specificity we assume that particles are confined to $z$ axis.

The particle-particle interaction consists of two contributions: a dipole-dipole interaction and a short-range repulsion.

The energy of dipole-dipole interaction is given by

\label{eq_dipole_dipole_interaction}
\begin{equation}
E = \frac{\mu_1 \mu_2}{r^3}[3 (\vec{m}_1 \cdot \vec{r}_{12})(\vec{m}_2 \cdot \vec{r}_{12}) - (\vec{m}_1 \cdot \vec{m}_2)]
\end{equation}
where $\mu_1 \vec{m}_1$ and $\mu_2 \vec{m}_2$ \textcolor{red}{ $\mu$ s are indexed because in general different particles can have different moment?} are dipole moments of interacting particles, $\vec{r}_{12}$ is the direction vector which connects particle centres, and $(\vec{a} \cdot \vec{b})$ denotes dot product.

By constraining particles in 1D tube we effectively enforce $\vec{r}_{12}$ to be co-aligned with $z$ axis. Defining energy in the units of particles dipole moment and assuming that $\mu_1 = \mu_2$, (\ref{eq_dipole_dipole_interaction}) can be simplified in the following way:

\label{eq_dipole_dipole_1D}
\begin{equation}
E_{attractive} = \frac{1}{r^3} [3 \cos \theta_1 \cos \theta_2 - (\vec{m}_1 \cdot \vec{m}_2)]
\end{equation}
\textcolor{red}{Yes, I've done dot product of directions to get angle between them, not the difference between $\theta$}
where $\theta_1$ and $\theta_2$ are the angles between particles dipole moments and $z$ axis for the first and second particle respectfully. Because of three-dimensionality of our model, we can't simplify the last term, which accounts for relative orientation of particles.

The repulsive part is described by Yukawa potential

\label{eq_yukawa_interaction}
\begin{equation}
E_{repulsive} = \frac{A \exp(-r k)}{r}
\end{equation}
where $A$ and $k$ -- parameters which describe particle properties (hardness of repulsion and particle size). \textcolor{red}{Should I provide here derivative in some point as function of A or k (which I assume will define hardness), and position of a minimum of a potential which accounts for the diameter?}

Summarizing all of the above, the particle-particle interaction potential is presented in the following form:
\label{eq_full_particle_particle_interraction}
\begin{equation}
E_{12} = -\left\{ \frac{\epsilon_{12} A \exp(-r k)}{r} +  \frac{\epsilon_{12}}{r^3} [3 \cos \theta_1 \cos \theta_2 - (\vec{m}_1 \cdot \vec{m}_2)]\right\}
\end{equation}
where $\epsilon_{12}$ \textcolor{red}{Why it's inconsistent with (1)? $\epsilon_{12} = \mu_1\mu_2$, is it not?} governs the interaction force, and through that --- temperature of the system. \textcolor{red}{maybe here I should put some figures?}

\section{Monte Carlo Simulations}

At first we concentrate on the equilibrium state of the system. For that we implement a common MC algorithm, each step consisting of test rotation and test move.

\textcolor{red}{Test rotation is evenly distributed over sphere random change of particle dipole moment direction.}

Test move is a simple change of a particle coordinate ($z_{t+1} = z_t + \Delta z$), where $\Delta z$ is a random value evenly distributed in $[+\delta, -\delta]$, whereas $\delta$ equals to \textcolor{red}{(actually, one half of a distance between a location of potential minimum and potential zero, but it's complicated)}. This constraint was done in order to eliminate possibility of particles switching positions by overstepping repulsive potential.

After move and rotation test step is evaluated with standard Metropolis acceptance criteria. For the sake of simplicity we restrict interactions to closest neighbours (particle $i$ interacts only with $i-1$ and $i+1$ particles) 

On each end of the system we implement continuous border conditions.

Important question is the length (number of test steps taken) of simulations (\textcolor{red}{Here will be steps picture?}).

\section{Production runs and results}

Our simulations have been made for a case of average density value ($\rho = 0.5$). As diameter of a particle we have taken the position of the minimum of (\ref{eq_full_particle_particle_interraction}), obtained when and $(\vec{m}_1 \cdot \vec{m}_2) = 0$ \textcolor{red}{see figure !}. The length of the simulation system $L = 2*N*d$, where $N$ -- number of particles and $d$ -- particle diameter.

In our survey we concentrate on the following \textcolor{red}{quantities??}.

Order parameter $S$ defined in standard way:
\begin{equation}
S = \frac{3 \overline{\cos^2 \theta} - 1}{2}
\end{equation}
where $\theta$ it the angle between particle direction and spatial axis, while horizontal bar denotes mean over all particles.

Orientation correlation as function of distance between particles:
\begin{equation}
\langle\cos \theta_1 \cos \theta_2\rangle \propto |z_2 - z_1|
\end{equation}

And particle chains length probability distribution \textcolor{red}{How to call it??}. One can think of different ways to define it, but the simplest is the following: two particles considered to be "chained" if 
\begin{equation}
\begin{cases}
	|z_1 - z_2| &\leq d\\
	\cos \theta_1 \cdot \cos \theta_2 &\geq 0
\end{cases}
\end{equation}

Second check enforces bonded particles to have same sign of projection of the dipole moment vector on the spatial axis. As we will show in the \textcolor{red}{Results} section, this is unnecessary, since on the equilibrium chained (by distance) particles tend to have the same orientation.


\bibliographystyle{plain}
\bibliography{Bibliography}
\end{document}