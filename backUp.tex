\section{BackUp - don't read!}
Our model is influenced by and meant to extend the model proposed in \cite{Marshall2015}. Main property of the mentioned model is the arising of ordered state in one dimension.

One-dimensional model in \cite{Marshall2015} consists of an oriented particles which have two different interaction patches A and B, which located on the opposite sides of a particle, and particles interacts if their patronymic patches are within interaction radius. Despite being confined in one dimension, particles can explore full three dimension orientation space.

The order parameter is defined as follows:
\begin{equation}
S = \frac{3 \overline{\cos^2 \theta} - 1}{2}
\end{equation}

Where $\theta$ it the angle between particle direction and spatial axis.

The intermolecular potential for particles of this type is given as

\begin{equation}
E(12) = E_{HS}(\vec{r}_{12}) - \varepsilon_{AB}(O_{AB}(12) + O_{BA}(12))
\end{equation}

where $1 = (z_1, \theta_1, \varphi_1)$ represents first particle position and orientation. In \cite{Marshall2015} the $\phi_{HS}(r_{12})$ is the hard sphere repulsion, which in 1D corresponds to

\begin{equation}
E_{HS}(\vec{r}_{12}) = \begin{cases}
	\infty &\text{$|z_1 - z_2| \leq d$}\\
   	0 	&\text{$|z_1 - z_2| > d$}
\end{cases}
\end{equation}

Since interaction is restricted to patronymic patches, the $O_{AB}$ is given by the following equation:

\begin{equation}
O_{AB}(12) = \begin{cases}
	1 &\text{$|z_1 - z_2| <= z_c$ and $\theta_{A1} \leq \theta_c$ and $\theta_{B2} \leq \theta_c $}\\
   	0 &\text{otherwise}
\end{cases}
\end{equation}

where $z_c$ and $\theta_c$ are the maximum distance between centres and maximum angle between particle orientation and spatial axis respectfully.

Being very simple to implement and still able to capture \emph{interesting???} results, this model have some flaws, main of which is discontinued particle-particle interaction function.

While being irrelevant for Monte-Carlo methods, this kind of interaction functions leads to infinite forces in Molecular Dynamics methods.

To remove discontinuity whereas saving important features (interaction anisotropy and hard sphere repulsion) of the model described above, we offer following modifications.
The intermolecular potential consists of repulsion and interaction parts.
\begin{equation}
E(12) = - \varepsilon_{AB} (E_{r} + E_{i})
\end{equation}

The repulsion part is given in form of Yukawa potential
\begin{equation}
E_{r} = \frac{A \exp(-r k)}{r}
\end{equation}
where $A$ and $k$ - parameters which describe particle properties. The interaction part is a simple electric dipole-dipole interaction
\begin{equation}
E_{i} = \frac{1}{r^3} [3 \cos \theta_1 \cos \theta_2 - \cos \delta \theta]
\end{equation}

As above, $\theta$ is the angle between particle orientation and spatial axis, whereas $\delta \theta$ is the angle between two particles orientations.
