%!TEX root = ..\main.tex
\section{Monte Carlo Simulations}

At first we concentrate on the equilibrium state of the system. For that purpose we implement a common MC algorithm in a following way.

Every test step we choose a particle at random. Than we subject the particle to the \emph{test move} and \emph{test rotation}.

\emph{Test move} is a particle displacement ($z_{t+1} = z_t + \delta z$), where $\delta z$ is a random value evenly distributed in $[+\eta, -\eta]$, with $\eta = 0.5 (\Delta z_{min} - \Delta z_0)$. Here $\Delta z_{min}$ is distance between two particles with co-aligned dipole moments on which the potential energy is minimal (eq. \eqref{eq:full_particle_particle_interraction} takes it's minimal value), and $\Delta z_0$ is the farthest distance between two co-aligned particles where potential energy is zero \textcolor{red}{is it sounds clear, or not?}.

\emph{Test rotation} consists of generating new orientation of the particle at random, evenly distributed over all orientation space.

After move and rotation test step is evaluated with standard Metropolis acceptance criteria. The move is accepted if
\begin{equation}
\label{eq:metropolis_acceptance}
	\eta \leq \exp(\epsilon^* \Delta E)
\end{equation}
where $\eta$ is a random number uniformly-distributed in $[0, 1)$, $\Delta E$ is change in system potential energy due to test step (rotation and translation), and $\epsilon^* = 1/k_B T$.

For the sake of simplicity we restrict interactions to the immediate neighbours (particle $i$ interacts only with $i-1$ and $i+1$ particles) 

On each end of the system we implement continuous border conditions.