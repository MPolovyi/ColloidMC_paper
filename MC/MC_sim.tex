\section{Monte Carlo Simulations}

We first study the equilibrium properties by means of Monte Carlo simulations.

At every Monte Carlo step we choose successively one particle starting from selected at random on first step, and going in positive direction of $z$ axis, and attempt first a translational and then a rotational move. In the translational move a particle is displaced along the z-axis ($z_{t+1} = z_t + \delta z$), where $\delta z$ is a random value uniformly distributed in the range $[+\eta, -\eta]$, with $\eta = 0.5 (\Delta z_{min} - \Delta z_0)$. Here $\Delta z_{min}$ is distance between two particles with co-aligned dipole moments on which the potential energy is minimal, and $\Delta z_0$ is the distance where potential energy becomes zero if we decrease distance starting from $\Delta z_{min}$ \textcolor{red}{In other words, we finding the minima, then we are moving to the left from minima, and as soon as energy becomes zero or above, we say that it's $\Delta z_0$. Off course I'm solving the equation numerically in the code, but it's the most straightforward way to explain I can come up with}. For the rotational move a new orientation is chosen uniformly at random. A move is accepted with probability $\eta < \mathrm{min} \left\{1, \, \exp(\epsilon^*\Delta E) \right\}$, where $\eta$ is a random number uniformly-distributed in $[0, 1)$, $\Delta E$ is change in system potential energy due to test step (rotation and translation), and $\epsilon^* = \epsilon/k_B T$.

For the sake of simplicity we restrict interactions to the two closest neighbors.

We considered periodic boundary conditions.