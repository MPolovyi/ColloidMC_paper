\documentclass[12pt, a4paper]{article}
\usepackage[utf8]{inputenc}
\usepackage[export]{adjustbox}
\usepackage{amsmath}
\usepackage{amsfonts}
\usepackage{amssymb}
\usepackage{xcolor}
\usepackage{graphicx}
\usepackage{subcaption}
\usepackage[left=2cm,right=2cm,top=2cm,bottom=2cm]{geometry}

\makeatletter
\renewcommand\thesubsection{\@arabic\c@subsection}
\makeatother

\newcommand{\figref}[1]{Figure~\ref{#1}}

\author{Maksym Polovyi}
\begin{document}

The phenomena under study is the existence of non-zero nematic order in a colloidal suspension consisting of short-rage interacting particles. As the model, we consider colloidal particles with a permanent dipole moment. While considering a three-dimensional system, we constrain particle translational motion to a one-dimensional line, i.e. particles are confined to the $z$ axis. However, particles can explore the full three dimensional orientation space.

This model can be used to describe for example the 1D hydrogen bonded wires in zeolites and carbon nanotubes.

In this work we concentrate on analysing equilibrium and dynamical properties of the model. We measure the nematic order of a system, and relative orientational  probability of the particles. To analyse the equilibrium properties, first, we perform Monte Carlo (MC) simulations, and then compare the obtained results with long-time Langevin Dynamics (LD) simulations. The comparison yields a good agreement for all reasonable values of simulation parameters (density and temperature), thus we assume that neither MC nor LD simulations were trapped in a local minima, and indeed reached an equilibrium configuration.

On the equilibrium, the system shows complete homogeneous structure, with particles aggregated in a clusters with a local orientational order. The dynamical evolution of all observed quantities proved to be an exponential over time relaxation to a respective equilibrium values.
The simulations of different systems sizes with other parameters being fixed, performed for MC and LD, hasn't shown any evidence of scaling.

\end{document}