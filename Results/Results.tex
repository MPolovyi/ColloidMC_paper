%!TEX root = ..\main.tex
\section{Results}
\subsection{Simulation details}
\label{subsec:simulation_details}

Defining $\rho = \frac{N D}{L}$ where $N$ --- number of particles and $D = \Delta z_{min}$ --- particle diameter calculated numerically, we immediately obtain the simulation system size $L = N D/ \rho$. In our simulations we considered cases of different densities $\rho = (0.25,\ 0.6,\ 0.95)$.

To check for existence of finite-size effects, for each value of density we perform simulations with different particle numbers $N = (1600,\ 3200)$. \textcolor{red}{I've already said that I do it for different density, should I say that I do it for different size instead?}

The Monte-Carlo simulations were allowed to equilibrate for $\Delta = 2 \cdot 10^5$ sweeps (one sweep consists of $N$ Monte-Carlo test steps), and after that every $\Delta$ sweeps were generated ensemble-averaged quantities, obtaining $10$ uncorrelated samples. To obtain initial configuration particles were distributed on a lattice with step $d = \frac{D}{\rho}$, and then were subjected to random displacement uniformly-distributed in $[-d/4, d/4]$. Initial orientation of particles were chosen uniformly at random.

The Langevin dynamics simulations were running with the following parameters: particle mass $m = 1$ \textcolor{red}{Units of particle mass}, fluid viscosity $\eta = 1$ \textcolor{red}{Units of some reference viscosity}, integration time step $d t = 0.005$ \textcolor{red}{Time units are usually obtained from diffusion, right?}. Initial particle coordinates were obtained as in the Monte-Carlo case. To check dependence of system evolution on the initial configuration, we made simulations staring from three possible orientation configurations. First was to select orientation uniformly at random, second was to have all particles oriented along the $z$ axis, and the third was to have particles with even index oriented along, and with odd index against $z$ axis. We will refer to this configurations as "random", "co-aligned" and "counter-aligned" respectively. The sampling was done directly --- ensemble averaging quantities over one simulation run we obtain one uncorrelated sample. \textcolor{red}{as long as the random generator is properly initialized}

We measured the following quantities:

Nematic order parameter $S$ is defined as:
\begin{equation}
\label{eq:nematic_order_parameter}
	S = \frac{3 \langle\cos^2 \theta\rangle - 1}{2}
\end{equation}
where $\theta$ it the angle between particle dipole moment and spatial axis, while angle brackets denotes ensemble average.

We also measured the orientation correlation as a function of the distance between particles:
\begin{equation}
	C(\Delta z) = \langle\cos \theta_1 \cos \theta_2\rangle \propto |\Delta z|
\end{equation}
where $\theta_1$ and $\theta_2$ are angles between spatial axis and dipole moments of particles which centers are separated by $\Delta z$ distance. \textcolor{red}{I do it average over some $\Delta z +/- \delta$ area. If I'll just write "separated by $\Delta z +/- \delta$ --- will it be clear?}

We measure the average chain length, where chain is a sequence of chained particles and two particles considered to be "chained" if 
\textcolor{red}{maybe here should be something about energy threshold, plus $<d$ maybe better to be $<d + \delta$?}
\begin{equation}
\begin{cases}
	|z_1 - z_2| &\leq d \\
	\cos \theta_1 \cdot \cos \theta_2 &\geq 0
\end{cases}
\end{equation}