%!TEX root = ..\main.tex
\section{Simulation details}
\label{sec:simulation_details}

Defining $\rho = \frac{N D}{L}$ where $N$ --- number of particles and $D = \Delta z_{min}$ --- particle diameter calculated numerically, we immediately obtain the simulation system size $L = N D/ \rho$. In our simulations we considered cases of different densities $\rho = (0.25,\ 0.5,\ 0.75)$.

To check for existence of finite-size effects, for each value of density we performed simulations with different system sizes $N = (1600,\ 3200,\ 6400)$. \textcolor{red}{I'm maintaining density as constant and changing $L$ and $N$ accordingly. I'm reporting here $N$ instead of $L$ because the particle diameter is $~0.62$, and for $N = 1600, \rho = 0.75$ the $L = 1322.667$, which for me is more confusing then number of particles}

The Monte-Carlo simulations were allowed to equilibrate for $\Delta = 3 \cdot 10^5$ Monte Carlo sweeps, and different quantities were measured after every $\Delta$ sweeps, obtaining $10$ uncorrelated samples. To generate the initial configuration, dipolar particles were regularly distributed along $z$ axis, with an interparticle distance $d = \frac{D}{\rho}$, and then displaced by a certain value, following a uniform distribution in the range $[-d/4, d/4]$. The orientation of the dipoles was generated uniformly at random.

The Langevin dynamics simulations were ran with the following parameters: unitary mass $m$ and viscosity $\mu$, and time integration step $d t = 0.01$. Initial particle coordinates were obtained as in the Monte-Carlo case. To check dependence of system evolution on the initial configuration, we performed simulations staring from three possible orientational configurations. First, we generated the initial orientations uniformly at random; second,  we aligned all particles along the $z$ axis; and third, particles of even index were oriented along, and with odd index against $z$ axis. We refer to these configurations as ``random'', ``co-aligned'' and ``counter-aligned'', respectively. The sampling was done directly --- ensemble averaging quantities over one simulation run, we obtain one uncorrelated sample.

We measured the following quantities:

Nematic order parameter $S$ is defined as:
\begin{equation}
\label{eq:nematic_order_parameter}
	S = \frac{3 \langle\cos^2 \theta\rangle - 1}{2}
	,
\end{equation}
where $\theta$ it the angle between particle dipole moment and $z$ axis, while angle brackets denotes ensemble average over all particles.

Next we measured probability distribution $p(\theta)$ of the angle $\theta$ between particle dipole moment and $z$ axis.

We also measured the orientation correlation as a function of the distance between particles:
\begin{equation}
\label{eq:distance_correlation}
	C(\Delta z) = \langle\cos \theta_i \cos \theta_j\rangle
	,
\end{equation}
where $\theta_i$ and $\theta_j$ are angles between spatial axis and dipole moments of particles for which $\Delta z - \delta < |z_j - z_i| \leq \Delta z + \delta$, where $2\delta$ is a predefined space sampling. Angle brackets denotes ensemble-averaging over all particles in all samples which satisfy distance criteria.

We measure the average chain length, where chain is a sequence of chained particles and two particles considered to be ``chained'' if 
\begin{equation}
\label{eq:chains_definition}
\begin{cases}
	|z_i - z_j| \leq d \\
	\theta_{i, j} \leq \alpha \text{ \textbf{or} } \theta_{i, j} \geq \pi - \alpha
\end{cases}
\end{equation}
where $d$ is a predefined separation distance after which particles are ``unchained'', and second condition makes particles to be oriented in the same direction and within a certain angle $\alpha$ along or against $z$ axis positive direction.

Using the same definition for a chain, we define one to be ``right'' chain if all the particles satisfy $\theta_i \leq \alpha$, ``left'' if $\theta_i \geq \pi - \alpha$, and ``undefined'' in all other cases. Then we can measure the probability of two neighbouring chains to be ``left-left'' or ``left-right'', etc.

For Langevin Dynamics we additionally measure autocorrelation function of a particle orientation:
\begin{equation}
\label{eq:autocorrelation_ld}
	C(t) = \langle\cos\theta_i(t) \cdot \cos\theta_i(0)\rangle
\end{equation}
here $\theta_i(0)$ is the orientation of a particle $i$ at the beginning of simulation and $\theta_i(t)$ is the orientation of the same particle in the same sample at the time $t$. The ensemble-averaging is done over all particles.