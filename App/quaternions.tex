\section{Appendix}
\subsection{Quaternions}

Quaternion $q = (\omega, \vec{a})$ with a norm $\norm{q} = 1$ defines rotation around axis parallel to $\vec{a}$ on the angle $\cos^{-1}(2 \omega)$. 

One can define quaternion norm
\begin{equation}
\norm{q} = |\omega| + \norm{\vec{a}}
\end{equation}

Quaternion conjugation
\begin{equation}
q^{-1} = (\omega, -\vec{a})
\end{equation}

And quaternion product
\begin{equation}
\label{eq:quaternion_product}
q_3 = q_1 q_2 = \left(\omega_1 \omega_2 - \vec{a}_1 \cdot \vec{a}_2, \quad \omega_1 \vec{a}_2 + \omega_2 \vec{a}_1 + [\vec{a}_1 \times \vec{a}_2] \right)
\end{equation}

If we have a unit vector $\vec{a}_1$ and want to rotate the vector $\vec{r}$  around it on the angle $\omega_1$, first we need to construct quaternion
\begin{equation}
q_1 = \left(
		\cos\left(\frac{\omega_1}{2}\right),
		\vec{a}_1 \sin\left(\frac{\omega_1}{2}\right)
	\right)
\end{equation}
which represents the rotation $\mathbb{R}_1$, and then to apply $\mathbb{R}_1$ to $\vec{r}$
\begin{equation}
\label{eq:quaternion_rotation}
	\vec{r}\,' = q_1\, \vec{r}\, q_1^{-1}
\end{equation}
where $\vec{r'}$ is the new state of the vector $\vec{r}$ after rotation, and $q_1^{-1}$ is the quaternion conjugate to $q_1$.

Multiplication $q_i \vec{r}$ should be done by the rules of quaternion product \eqref{eq:quaternion_product}. The vector $\vec{r}$ should be treated as quaternion $q_r = (0, \vec{r}\,)$.

If we have a rotation $\mathbb{R}_1$ represented by quaternion $q_1$ and rotation $\mathbb{R}_2$ represented by quaternion $q_2$, and we want to apply them to vector $\vec{r}$ in order $\mathbb{R}_1,\ \mathbb{R}_2$, by definition \eqref{eq:quaternion_rotation} we can write
\begin{equation}
	\vec{r}\,''
	= q_2\, \vec{r}\,' q_2^{-1} 
	= q_2\, q_1\, \vec{r}\, q_1^{-1}\, q_2^{-1}
\end{equation}
and if we define $q_3 = q_2\, q_1$, then $q_3^{-1} = q_1^{-1}\, q_2^{-1}$ and then
\begin{equation}
	\vec{r}\,'' = q_3\, \vec{r}\, q_3^{-1}
\end{equation}
Therefore, compound rotation of two sequential rotations which are represented by quaternions $q_1$ and $q_2$ is represented by quaternion $q_3 = q_2\, q_1$, where $q_2\, q_1$ is quaternion product \eqref{eq:quaternion_product}. Rotation represented by the rightmost quaternion is applied first, then the one represented by the second-rightmost, and so on.

If we define a global reference frame, then any quaternion $q$ can also define a rotated reference frame. To get coordinates $\vec{r}\,'$ of a vector $\vec{r}$ defined in global frame in rotated, one only have to apply quaternion $q$ to vector $\vec{r}$. To go back to global reference frame from $\vec{r}\,'$, one have to apply $q^{-1}$ to $\vec{r}\,'$.
\begin{equation}
	\begin{aligned}
		\vec{r}\,' = q \vec{r} q^{-1}
	\end{aligned}
	\qquad
	\qquad
	\begin{aligned}
		\vec{r} = q^{-1} \vec{r}\,' q
	\end{aligned}
\end{equation}