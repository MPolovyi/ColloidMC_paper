\section{Appendix}
\subsection{Quaternions}

Quaternion $q = (\omega, \boldsymbol{a})$ with a norm $\norm{q} = 1$ defines rotation around axis parallel to $\boldsymbol{a}$ on the angle $\cos^{-1}(2 \omega)$. 

One can define quaternion norm
\begin{equation}
\norm{q} = |\omega| + \norm{\boldsymbol{a}}
\end{equation}

Quaternion conjugation
\begin{equation}
q^{-1} = (\omega, -\boldsymbol{a})
\end{equation}

And quaternion product
\begin{multline}
\label{eq:quaternion_product}
q_3 = q_1 q_2 \\ = \left(\omega_1 \omega_2 - \boldsymbol{a}_1 \cdot \boldsymbol{a}_2, \quad \omega_1 \boldsymbol{a}_2 + \omega_2 \boldsymbol{a}_1 + [\boldsymbol{a}_1 \times \boldsymbol{a}_2] \right)
\end{multline}

If we have a unit vector $\boldsymbol{a}_1$ and want to rotate the vector $\boldsymbol{b}$  around it on the angle $\omega_1$, first we need to construct quaternion
\begin{equation}
q_1 = \left(
		\cos\left(\frac{\omega_1}{2}\right),
		\boldsymbol{a}_1 \sin\left(\frac{\omega_1}{2}\right)
	\right)
\end{equation}
which represents the rotation $\mathbb{R}_1$, and then to apply $\mathbb{R}_1$ to $\boldsymbol{b}$
\begin{equation}
\label{eq:quaternion_rotation}
	\boldsymbol{b}\,' = q_1\, \boldsymbol{b}\, q_1^{-1}
\end{equation}
where $\boldsymbol{r'}$ is the new state of the vector $\boldsymbol{b}$ after rotation, and $q_1^{-1}$ is the quaternion conjugate to $q_1$.

Multiplication $q_i \boldsymbol{b}$ should be done by the rules of quaternion product \eqref{eq:quaternion_product}. The vector $\boldsymbol{b}$ should be treated as quaternion $q_r = (0, \boldsymbol{b}\,)$.

If we have a rotation $\mathbb{R}_1$ represented by quaternion $q_1$ and rotation $\mathbb{R}_2$ represented by quaternion $q_2$, and we want to apply them to vector $\boldsymbol{b}$ in order $\mathbb{R}_1,\ \mathbb{R}_2$, by definition \eqref{eq:quaternion_rotation} we can write
\begin{equation}
	\boldsymbol{b}\,''
	= q_2\, \boldsymbol{b}\,' q_2^{-1} 
	= q_2\, q_1\, \boldsymbol{b}\, q_1^{-1}\, q_2^{-1}
\end{equation}
and if we define $q_3 = q_2\, q_1$, then $q_3^{-1} = q_1^{-1}\, q_2^{-1}$ and then
\begin{equation}
	\boldsymbol{b}\,'' = q_3\, \boldsymbol{b}\, q_3^{-1}
\end{equation}
Therefore, compound rotation of two sequential rotations which are represented by quaternions $q_1$ and $q_2$ is represented by quaternion $q_3 = q_2\, q_1$, where $q_2\, q_1$ is quaternion product \eqref{eq:quaternion_product}. Rotation represented by the rightmost quaternion is applied first, then the one represented by the second-rightmost, and so on.

If we define a global reference frame, then any quaternion $q$ can also define a rotated reference frame. To get coordinates $\boldsymbol{b}\,'$ of a vector $\boldsymbol{b}$ defined in global frame in rotated, one only have to apply quaternion $q$ to vector $\boldsymbol{b}$. To go back to global reference frame from $\boldsymbol{b}\,'$, one have to apply $q^{-1}$ to $\boldsymbol{b}\,'$.
\begin{equation}
	\begin{aligned}
		\boldsymbol{b}\,' = q \boldsymbol{b} q^{-1}
	\end{aligned}
	\qquad
	\qquad
	\begin{aligned}
		\boldsymbol{b} = q^{-1} \boldsymbol{b}\,' q
	\end{aligned}
\end{equation}