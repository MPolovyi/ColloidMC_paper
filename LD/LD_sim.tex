%!TEX root = ..\main.tex
\section{Langevin Dynamics}
\label{sec:langevin_dynamics}

\subsection{Langevin equations}
\label{subsec:langevin_equations}
To study dynamics we performed Langevin dynamics simulations. Accordingly, each particle trajectory is resolved by solving two coupled stochastic differential equations,
\begin{equation}
\label{eq:langevin_theory_1}
	\dot{\boldsymbol{r}}(t) = \boldsymbol{v}(t)
\end{equation}
and
\begin{equation}
\label{eq:langevin_theory_2}
	m \dot{\boldsymbol{v}}(t) = \boldsymbol{f}(\boldsymbol{r}, t) - \alpha_{tr} \boldsymbol{v}(t) + \boldsymbol{\beta}_{tr}(t)
	,
\end{equation}
where $m$ is the particle mass, $\boldsymbol{r}$ and $\boldsymbol{v}$ are the particle coordinates and velocity,  $\alpha_{tr}$ is the damping coefficient related to the drag, and $f(r, t)$ is a force acting on a particle, which comprises of interparticle interaction. $\boldsymbol{\beta}_{tr}$ is a stochastic force, and in order to satisfy dissipation-fluctuation theorem, every component of it is assumed to be Gaussian-distributed, with
\begin{equation}
\langle\beta_{tr}(t)\rangle = 0
\end{equation}
and
\begin{equation}
\label{eq:stochastic_term_dispersion}
	\langle\beta_{tr}(t)\beta_{tr}(t')\rangle = 2 \alpha_{tr} k_B 		T\delta(t - t')
	,
\end{equation}
where $k_B$ is the Boltzmann constant, $T$ is thermostat temperature, $\delta$ is the Dirac delta function and the average is an ensemble average.

Similarly, the trajectory of rotation motion is obtained by solving two stochastic differential equations,
\begin{equation}
\label{eq:langevin_theory_3}
	\dot{\boldsymbol{\phi}}(t) = \boldsymbol{\omega}(t)
\end{equation}
and
\begin{equation}
\label{eq:langevin_theory_4}
	m \dot{\boldsymbol{\omega}}(t) = \boldsymbol{\tau}(r, t) - \alpha_{r} \boldsymbol{\omega}(t) + \boldsymbol{\beta}_{r}(t)
	,
\end{equation}
where $\phi$ is particle orientation angles, $\omega$ is the angular velocity, $I$ is the inertia and $\tau$ is the torque exerted on a particle. $\boldsymbol{\beta}_{r}(t)$ is the torque resulting from stochastic force rotating particle, and its components are also assumed to be Gaussian-distributed with zero mean.

As we can see from Eqs.~(\ref{eq:langevin_theory_1}, \ref{eq:langevin_theory_2}, \ref{eq:langevin_theory_3}, \ref{eq:langevin_theory_4}), for the rotational motion the form of equations remains unchanged, and they are equivalent to the translation equations under following transformation.
\begin{equation}
\label{eq:rotation_translation_substitution}
	\begin{aligned}[c]
		\boldsymbol{\phi} &\rightarrow \boldsymbol{r}        \\
		\boldsymbol{\omega} &\rightarrow \boldsymbol{v} 
	\end{aligned}
	\qquad
	\qquad
	\begin{aligned}[c]
		I &\rightarrow m        \\
		\boldsymbol{\tau}(r, t) &\rightarrow \boldsymbol{f}(r, t)
	\end{aligned}
\end{equation}

The scalar values are transformed as is, while vectors are replaced with pseudo-vectors.

The damping coefficients $\alpha$ for translation ($\alpha_{tr}$) and rotation ($\alpha_r$) are different, and can be obtained in following way.

First to simplify the particle-medium interaction we need to define damping time $\tau$, which determines how rapidly thermal fluctuations in particles positions and orientations decays, and for translational motion it is given by
\begin{equation}
\label{eq:Translation_damping_time}
	\tau_{tr} = \frac{m}{6 \pi \eta R}
	,
\end{equation}
where $m$ is the particle mass, $\eta$ is the viscosity and $R$ is the particle radius. For rotation $\tau_{tr} / \tau_{r} = 10/3$, which can be obtained from Stokes-Einstein-Debye relations and relation between particle mass and inertia.

And then the $\alpha_t$ and $\alpha_r$ are defined
\begin{equation}
	\begin{aligned}
		\alpha_{tr} = \frac{m}{\tau_{tr}}
	\end{aligned}
	\qquad
	\qquad
	\begin{aligned}
		\alpha_r = \frac{I}{\tau_r}
	\end{aligned}
	,
\end{equation}
where $m$ is the particle mass, $I$ is the inertia, $\tau_{tr,r}$ are the damping times.

\subsection{Interactions}

To perform simulations we need to define forces acting on the particles.

First of all, as in the Monte-Carlo simulations we restrict interactions to the immediate neighbors (particle $i$ interacts only with particles $i+1$ and $i-1$).

The force acting on a particle is defined by gradient of potential energy.
\begin{equation}
\label{eq:full_force}
	\boldsymbol{F}_{ij}
		= -\boldsymbol{\nabla} E_{ij}
		=  \frac{\hat{r}}{r^4} \left[3 D - A\, r^2 (k r +1) \, \exp(-k r) \right],
\end{equation}
where $D = 3 \cos \theta_1 \cos \theta_2 - (\hat{m}_1 \cdot \hat{m}_2)$ stands for orientational part of dipole-dipole potential and $\boldsymbol{r}$ connects particle centers. The other parameters are the same as in Eqs.~\eqref{eq:dipole_dipole_1D}~and~\eqref{eq:yukawa_interaction}.

The torque on a particle results only from the dipole-dipole interaction. Therefore it is calculated as
\begin{equation}
\label{eq:dipole_torque}
	\boldsymbol{\tau}  = \mu[\hat{\mu} \times \boldsymbol{E}_d ],
\end{equation}
where $\boldsymbol{\mu}$ is dipole moment of particle on which torque acts, and $\boldsymbol{E}_d$ is dipole field produced by acting particle:

\begin{equation}
\label{eq:dipole_field}
	\boldsymbol{E}_d = \frac{\mu}{r^3}
		\left(3 (\hat{\mu} \cdot \hat{r}) \hat{r} - \hat{\mu} \right),
\end{equation}
where $\mu$ is dipole moment of the particle inducing the field.

We integrate the above equations using the Velocity-Verlet integration scheme~\cite{Taylor2013}. The adaptation of the scheme to use quaternions to define rotation dynamic follows the framework expressed in~\cite{Zhao2013}. The gist of adaptation is presented in Appendices.

\subsection{Diffusion properties}
For non-interacting spheres the relation between rotational and translational diffusion coefficient can be obtained \cite{C5SM02754C}. The former is given by the Stokes-Einstein relation
\begin{equation}
\label{eq:translational_diffusion_coefficient}
	D_{tr} = \frac{k_B T}{6 \pi \eta R}
\end{equation}
and the latter is given by Stokes-Einstein-Debye relation
\begin{equation}
\label{eq:rotational_diffusion_coefficient}
	D_r = \frac{k_B T}{8 \pi \eta R^3}
	,
\end{equation}
where $T$ is the thermostat temperature, $\eta$ the viscosity of the medium and $k_B$ the Boltzmann constant. The relation between $D_{tr}$ and $D_r$ is given by
\begin{equation}
	\frac{D_r}{D_{tr}} = \frac{3}{4 R^2}
	.
\end{equation}

Using well known estimates for diffusion coefficients when $t \rightarrow \inf$, we can check the implementation and quality of integration scheme. The results are shown in the Appendices.
