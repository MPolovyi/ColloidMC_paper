%!TEX root = ..\main.tex
\section{Relaxation Dynamics}
\label{sec:langevin_dynamics}

\subsection{Langevin equations}
\label{subsec:langevin_equations}
To study the relaxation dynamics, using Langevin dynamics we performed particle-based simulations. Each particle trajectory is resolved by solving two coupled stochastic differential equations,
\begin{equation}
\label{eq:langevin_theory_1}
	\dot{\boldsymbol{r}}(t) = \boldsymbol{v}(t)
\end{equation}
and
\begin{equation}
\label{eq:langevin_theory_2}
	m \dot{\boldsymbol{v}}(t) = \boldsymbol{f}(\boldsymbol{r}, t) - \frac{m}{\tau_{tr}} \boldsymbol{v}(t) + \boldsymbol{\beta}_{tr}(t)
	,
\end{equation}
where $m$ is the particle mass, $\boldsymbol{r}$ and $\boldsymbol{v}$ are the particle coordinates and velocity,  $\tau_{tr}$ is the translational damping time, and $\boldsymbol{f}(\boldsymbol{r}, t)$ is the net force acting on a particle. $\boldsymbol{\beta}_{tr}$ is a stochastic force, every component of which is Gaussian-distributed, in order to satisfy dissipation-fluctuation theorem, with
\begin{equation}
\langle\boldsymbol{\beta}_{tr}(t)\rangle = 0
\end{equation}
and
\begin{equation}
\label{eq:stochastic_term_dispersion}
	\langle\boldsymbol{\beta}_{tr}(t)\boldsymbol{\beta}_{tr}(t')\rangle = 2 \frac{m}{\tau_{tr}} k_B 		T\delta(t - t')
	,
\end{equation}
where $k_B$ is the Boltzmann constant, $T$ is thermostat temperature, $\delta$ is the Dirac delta function and the average is an ensemble average.

Similarly, the equations for the rotational degrees of freedom are
\begin{equation}
\label{eq:langevin_theory_3}
	\dot{\boldsymbol{\phi}}(t) = \boldsymbol{\omega}(t)
\end{equation}
and
\begin{equation}
\label{eq:langevin_theory_4}
	m \dot{\boldsymbol{\omega}}(t) = \boldsymbol{\tau}(r, t) - \frac{m}{\tau_{r}} \boldsymbol{\omega}(t) + \boldsymbol{\beta}_{r}(t)
	,
\end{equation}
where $\phi$ is particle orientation angles, $\boldsymbol{\omega}$ is the angular velocity, $I$ is the inertia and $\boldsymbol{\tau}$ is the torque exerted on a particle. $\boldsymbol{\beta}_{r}(t)$ is the a stochastic torque, and its components are also assumed to be Gaussian-distributed with zero mean.

As we can see from Eqs.~\eqref{eq:langevin_theory_1}, \eqref{eq:langevin_theory_2}, \eqref{eq:langevin_theory_3}, \eqref{eq:langevin_theory_4}, for the rotational motion the form of equations remains unchanged, and they are equivalent to the translation equations under following transformation:
\begin{equation}
\label{eq:rotation_translation_substitution}
	\begin{aligned}[c]
		\boldsymbol{\phi} &\rightarrow \boldsymbol{r}        \\
		\boldsymbol{\omega} &\rightarrow \boldsymbol{v} 
	\end{aligned}
	\qquad
	\qquad
	\begin{aligned}[c]
		I &\rightarrow m        \\
		\boldsymbol{\tau}(\boldsymbol{r}, t) &\rightarrow \boldsymbol{f}(\boldsymbol{r}, t)
	\end{aligned}
\end{equation}
Considering spherical particles, for translation motion the damping time is given by
\begin{equation}
\label{eq:Translation_damping_time}
	\tau_{tr} = \frac{m}{6 \pi \eta R}
	,
\end{equation}
where $m$ is the particle mass, $\eta$ is the viscosity and $R$ is the particle radius. From Stokes-Einstein-Debye relation~\cite{MazzaPhysRevE76031203},
\begin{equation}
	\frac{D_r}{D_{tr}} = \frac{3}{4 R^2}
	,
\end{equation}
we find the rotational damping time $\tau_{r} = 3\tau_{tr} / 10$.

\subsection{Pairwise interactions}

As in the Monte-Carlo simulations we restrict interactions to the immediate neighbors.

The force acting on a particle is defined by gradient of potential energy.
\begin{equation}
\label{eq:full_force}
	\boldsymbol{F}_{ij}
		= -\boldsymbol{\nabla} U_{ij},
\end{equation}
where $U_{ij} = U^\mathrm{dip}_{ij} + U^\mathrm{rep}_{ij}$ as defined by Eqs.~\eqref{eq:dipole_dipole_1D}~and~\eqref{eq:yukawa_interaction}.

The torque on a particle results only from the dipole-dipole interaction. Therefore it is calculated as
\begin{equation}
\label{eq:dipole_torque}
	\boldsymbol{\tau}  = \mu[\hat{\mu} \times \boldsymbol{E}_d ],
\end{equation}
where $\boldsymbol{\mu}$ is the dipole moment of particle on which the torque acts, and $\boldsymbol{E}_d$ is the dipole field produced by acting particle:
\begin{equation}
\label{eq:dipole_field}
	\boldsymbol{E}_d = \frac{\mu}{r^3}
		\left(3 (\hat{\mu} \cdot \hat{r}) \hat{r} - \hat{\mu} \right),
\end{equation}
where $\boldsymbol{\mu}$ is dipole moment of the particle inducing the field.

We integrate the above equations using the Velocity-Verlet integration scheme~\cite{Taylor2013}. The adaptation of the scheme to use quaternions to define rotation dynamic follows the framework described in Ref.~\onlinecite{Zhao2013}. Further details are provided in the Appendices.

%\subsection{Diffusion properties}
%For non-interacting spheres the relation between rotational and translational diffusion coefficient can be obtained \cite{C5SM02754C}. The former is given by the Stokes-Einstein relation
%\begin{equation}
%\label{eq:translational_diffusion_coefficient}
%	D_{tr} = \frac{k_B T}{6 \pi \eta R}
%\end{equation}
%and the latter is given by Stokes-Einstein-Debye relation
%\begin{equation}
%\label{eq:rotational_diffusion_coefficient}
%	D_r = \frac{k_B T}{8 \pi \eta R^3}
%	,
%\end{equation}
%where $T$ is the thermostat temperature, $\eta$ the viscosity of the medium and $k_B$ the Boltzmann constant. The relation between $D_{tr}$ and $D_r$ is given by
%\begin{equation}
%	\frac{D_r}{D_{tr}} = \frac{3}{4 R^2}
%	.
%\end{equation}
%
%Using well known estimates for diffusion coefficients when $t \rightarrow \inf$, we can check the implementation and quality of integration scheme. The results are shown in the Appendices.
